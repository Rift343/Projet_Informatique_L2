\documentclass[a4paper, 12pt]{article}
\usepackage[french]{babel}
\usepackage{libertine}
\usepackage{setspace}
\usepackage[pdftex]{graphicx}
\usepackage{fullpage}
\usepackage{hyperref}
\usepackage{listings}
\usepackage{graphicx}
\usepackage{fancyhdr}
% Une ligne commentaire débute par le caractère « % »

\documentclass[a4paper]{article}

% Options possibles : 10pt, 11pt, 12pt (taille de la fonte)
%                     oneside, twoside (recto simple, recto-verso)
%                     draft, final (stade de développement)

\usepackage[utf8]{inputenc}   % LaTeX, comprends les accents !
\usepackage[T1]{fontenc}      % Police contenant les caractères français 


\usepackage[a4paper,left=2cm,right=2cm]{geometry}% Format de la page, réduction des marges
\usepackage{graphicx}  % pour inclure des images

%\pagestyle{headings}        % Pour mettre des entêtes avec les titres
                              % des sections en haut de page

 \title{         % Les paramètres du titre : titre, auteur, date
  Projet de programmation}          
\author{Groupe \emph{XXX}\\
  \emph{François, Paul et les autres}\\
    L2 informatique\\
  Faculté des Sciences\\
Université de Montpellier.}
        


\begin{document}
\centerline{\Huge\bf HAI405I}
\vspace*{1.5cm}
\begin{center}               % pour centrer 
	
	\framebox[8cm]{
  %\includegraphics[width=10cm]{logo.pdf}   % insertion d'une image
	ici un logo si vous le souhaitez.
	}

\end{center}
\vspace*{1.5cm}

\fbox{\centerline{\Huge Projet de programmation}}

\vspace*{1.5cm}

\noindent{\large\bf Groupe XXX :}

\begin{itemize}
\item toto 1
\item toto 2
\item toto 3
\item toto 4
\end{itemize}
\vspace*{1.5cm}
\begin{center}
  L2 informatique\\
  Faculté des Sciences\\
Université de Montpellier.
\end{center}

\newpage
Le nombre de pages s'entend en police de taille 11, sans compter les figures tout en restant lisible.

\section{Organisation}

Organisation et répartition du travail au cours du temps et dans l'équipe incluant les outils de collaboration utilisés  (environ une page).

\section{Cahier des charges}

Présentation succinte  de ce qui a été fait/pas fait/buggué par rapport au cahier des charges initiales et éventuellement extensions (environ une page).

\section{Architecture et choix techniques}

1 à 2 page(s) 
\begin{itemize}
\item Architecture du projet
\item Choix technologiques  avec leurs justifications (pourquoi ce choix, comment il a été fait, par qui...),
\end{itemize}

    
\section{Détail}

Explication technique de deux ou trois points (1 à 3 pages max)
\begin{description}
\item[Obligatoire] Comment se fait la récupération des réponses et l'affichage en direct sur le poste enseignant,
\item[Au choix] Un ou deux  points techniques supplémentaires.  
\end{description}

\section{Bilan}

Bilan et analyse rétrospective et difficultés rencontrées (environ une page). Ce bilan peut aborder à la fois les aspects organisationnels et techniques.

    \end{document}


