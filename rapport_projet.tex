\documentclass[a4paper, 12pt]{article}
\usepackage[french]{babel}
\usepackage{libertine}
\usepackage{setspace}
\usepackage[pdftex]{graphicx}
\usepackage{fullpage}
\usepackage{hyperref}
\usepackage{listings}
\usepackage{graphicx}
\usepackage{fancyhdr}
% Une ligne commentaire débute par le caractère « % »

\documentclass[a4paper]{article}

% Options possibles : 10pt, 11pt, 12pt (taille de la fonte)
%                     oneside, twoside (recto simple, recto-verso)
%                     draft, final (stade de développement)

\usepackage[utf8]{inputenc}   % LaTeX, comprends les accents !
\usepackage[T1]{fontenc}      % Police contenant les caractères français 


\usepackage[a4paper,left=2cm,right=2cm]{geometry}% Format de la page, réduction des marges
\usepackage{graphicx}  % pour inclure des images

%\pagestyle{headings}        % Pour mettre des entêtes avec les titres
                              % des sections en haut de page

 \title{         % Les paramètres du titre : titre, auteur, date
  Projet de programmation}          
\author{Groupe \emph{XXX}\\
  \emph{François, Paul et les autres}\\
    L2 informatique\\
  Faculté des Sciences\\
Université de Montpellier}
        


\begin{document}
\centerline{\Huge\bf HAI405I}
\vspace*{1.5cm}
\begin{center}               % pour centrer 
	
	
  \includegraphics[width=5cm]{logo-fds.png}   % insertion d'une image
	

\end{center}
\vspace*{1.5cm}

\fbox{\centerline{\Huge Projet de programmation}}

\vspace*{1.5cm}

\noindent{\Large\bf Groupe 20 :}\\ \\

\begin{itemize}\large
\item Chevalier Clement\large
\item Conrath Matthieu\large
\item Pavie--Routaboul Clément\large
\item Rebagliato Lucas\large
\end{itemize}
\\
\\
\vspace*{1.5cm}
\begin{center}
  L2 informatique\\
  Faculté des Sciences\\
Université de Montpellier.
\end{center}

\newpage

\section{Organisation}

\subsection{Outils}
Pour notre projet on a utiliser différent outils. Tout d'abord pour on a utiliser un espace de dépot pour notre code. Etant donné que tout les membres du groupe connaissaient et s'avait utilisé Github avant même le début du projet, Github à donc été choisit intuitivement. Ensuite vient le moment des logiciels sur lequelle nous allions travailler. Etant donné que l'on était habitué à travailler avec Visual Studio Code et VScodium alors ce choix à aussi été naturelle. En plus Visual Studio Code et VScodium possède tout deux une intégration de github et de python ce qui facilite l'écriture ainsi que la mise en commun des travaux.

ajout de figure pours vscode et github
\subsection{Organisation du travail}
Nous nous somme au départ du second projet répartit la travail de la facon suivante: Clément Chevalier s'occuperait dans un premier temps de la gestion des étiquette, Rebagliato Lucas et Pavie--Routaboul Clément s'occuperai des séquences des questions et des comptes étudiants, ils seront rejoint par Clément Chevalier lorsque qu'il aurra terminé ca partie. Et Conrath Matthieu s'occupera de l'écriture et de la récupération des donneés dans les csv(historique et enregistrement des étudiants).
\subsection{Déroulement du projet dans le temps}
Ainsi l'on commenca rapidement par réaliser la création des comptes étudiants. Conrath Matthieu s'occupa de l'enregistrement, la récupération et la modification des comptes étudiants. Dans le même temps Pavie--Routaboul Clément et Rebagliato Lucas s'occuperent de mettre en place la récupération d'un fichier csv ainsi que de la connexion d'un étudiant à son compte. Ensuite Conrath Matthieu à réaliser l'écriture et la lecture des fichiers d'historique pendant que Pavie--Routaboul Clément et Rebagliato Lucas s'occuperent des séquences et des questions en direct. De son côté Chevalier Clément corriga la manière dont les étiquettes était crée lors de la création des questions.   

\section{Cahier des charges}

Avant de commencer l'on doit rappeler ce qui a été fait durant la première étape du projet. Nous avons permis la création de compte ainsi que la création et la visulisation de questions et de feuilles de question. Cepandent nous avions pas permis à l'utilisateur d'utiliser les étiquettes déja existantes. Ainsi la première étape à faire et des réglers ce petit problème.

\section{Architecture et choix techniques}

1 à 2 page(s) 
\begin{itemize}
\item Architecture du projet
\item Choix technologiques  avec leurs justifications (pourquoi ce choix, comment il a été fait, par qui...),
\end{itemize}

    
\section{Détail}

Explication technique de deux ou trois points (1 à 3 pages max)
\begin{description}
\item[Obligatoire] Comment se fait la récupération des réponses et l'affichage en direct sur le poste enseignant,
\item[Au choix] Un ou deux  points techniques supplémentaires.  
\end{description}

\section{Bilan}

Bilan et analyse rétrospective et difficultés rencontrées (environ une page). Ce bilan peut aborder à la fois les aspects organisationnels et techniques.

    \end{document}
